\enquote{}
\texteuro\
\textbf{}
\newpage
\vspace*{2cm}
Verweise im Text: \cite{doc:stz} und \cite{doc:gun}.
\underline{}


\begin{figure}[h]
\centering
\includegraphics[width=\textwidth]{fig/PP/ZGW_Sternkoppler_crop}
\caption{ZGW-Netzwerk - Beispiel 1}
\label{fig:ddd}
\end{figure}



\begin{figure}[h]
\centering
\includegraphics[width=10cm]{fig/ZGW/ZGW_cut_white.jpg}
\caption{Zentrales Gateway}
\label{fig:zgw_bild}
\end{figure}



\begin{wrapfigure}{r}{7cm}
	\vspace{-20pt}
  \begin{center}
    \includegraphics[width=4cm]{fig/PP/ZGW_Sternkoppler_crop}
  \end{center}
  \vspace{-20pt}
  \caption{ZGW Sternkoppler}
  \label{fig:ZGW_Sternkoppler_crop}
  \vspace{-10pt}
\end{wrapfigure}



\begin{itemize}
    \item acht Bauern
    \item zwei T�rme dsfffdfddsdf dsfd sf ds fsdf   dsfdsf df ds fds f dsf d f d fd sf ds f dfds fsd f dsf d f sdf sd f
    sdfdsfddfds  dsf ds fds f ds ds f sd fd f dsf ds f
    \item zwei Springer
    \item zwei L�ufer
    \item eine Dame
    \item ein K�nig
  \end{itemize}




\begin{enumerate}
    \item 
    \item sdfs
    \item sdf
    \item sdf\#
\end{enumerate}



\begin{description}
   \item[\textbf{}] 
   \item[\textbf{}] 
   \item[\textbf{}] 
\end{description}



\begin{figure}
\centering
\lstset{language=[Sharp]C}
\begin{lstlisting}[frame=tlrb] %, caption={Hallo Welt}, linewidth=\textwidth] %caption={Hallo Welt},linewidth=\textwidth
XmlDocument doc1 = new XmlDocument();
doc1.Load(filename); //Einlesen der Datei
XmlElement root = doc1.DocumentElement;
XmlNodeList frames = root.GetElementsByTagName("fx:FRAME");
foreach (XmlNode frame in frames)
{
	tbl_frame.FrameID = frame.Attributes.GetNamedItem("ID").Value; //Auslesen der Frame-ID
}
	
fibex4bmw = FIBEX.Load(filename);
foreach (var frame in fibex4bmw.ELEMENTS.FRAMES.FRAME)
{
	tbl_frame.FrameID = frame.ID;
}
\end{lstlisting}
\caption{C-Code2}
\label{fig:CCode2}
\end{figure}



\chapter{Konzeption und Realisierung}
\label{sec:kur}


\section{Anforderung}
\label{sec:kur-a}

\subsection{Konzept}
\label{sec:kur-k}

\subsubsection{Realisierung - C}
\label{sec:real-unterc}



%\begin{table} [h]
%	\tablestylecommon
%    \begin{tabularx}{\textwidth}{ | l | l | l | l | X |} %p{5cm} \begin{tabularx}{\textwidth}{l X}
%
%    \hline        
%    \tablehead \tableheadcolor Pin-Nr. & \tablehead \tableheadcolor Signal & \tablehead \tableheadcolor Funktion %
%     & \tablehead \tableheadcolor Abschluss-widerstand & \tablehead \tableheadcolor Bemerkung \\ \hline
%    
%    \tablebody
%    44 & CAN-4_H & \multirow{2}{*}{CAN-4 [Highspeed] (D-CAN)} & \multirow{2}{*}{2x68ohm} & \multirow{2}{*}{Im SG terminiert (bis 1 MBit)} \\ \cline{1-2}
%    45 & CAN-4_L &  &  &  \\ \hline
%    
%    \tablebody
%    44 & CAN-4_H & \multirow{2}{*}{CAN-4 [Highspeed] (D-CAN)} & \multirow{2}{*}{2x68ohm} & \multirow{2}{*}{Im SG terminiert (bis 1 MBit)} \\ \cline{1-2}
%    45 & CAN-4_L &  &  &  \\ \hline
%    
%      
%   
%    
%    \end{tabularx}
%	\caption{Pinbelegung des Zentralen Gateways}
%	\label{tab:pin_zgw}
%\end{table} 