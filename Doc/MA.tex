%%%%%%%%%%%%%%%%%%%%%%%%%%%%%%%%%%%%%%%%%%%%%%%%%%%%%%%%%%%%%%%%%%%%%%%%%%%
%
% Bachelors Thesis
% Author: Ingo von Bargen
%
%%%%%%%%%%%%%%%%%%%%%%%%%%%%%%%%%%%%%%%%%%%%%%%%%%%%%%%%%%%%%%%%%%%%%%%%%%%


% % Preamble BEGINN %%%%%%%%%%%%%%%%%%%%%%%%%%%%%%%%%%%%%%%%%%%%%%%%%%%%%%%%%




%%% Preamble (Pakete)
%\input{preamble/pre-packages}

%%% Neue Befehle
%\input{preamble/pre-newcommands}
%\input{preamble/pre-tablecommands} % Fuer Tabellen

%%% Silbentrennung
%% ------------------------------------------------------------------------
% LaTeX - Preambel ******************************************************
% ------------------------------------------------------------------------
% pre-hyphenation
% ========================================================================
\hyphenation{Ausgabe-format}
\hyphenation{Ros-Java}
\hyphenation{Ros-Java-Core}
\hyphenation{Ros-Android}
\hyphenation{Ros-Android-Core}
\hyphenation{Toll-kette}
\hyphenation{ROS-Text-View}
\hyphenation{ADVISORS}
\hyphenation{Ros-Master}
\hyphenation{pub-lish}
\hyphenation{sub-scribe}


%\bibliography{bib/bib.bib}
%\addbibresource{bib/literatur.bib}

% % Preamble ENDE %%%%%%%%%%%%%%%%%%%%%%%%%%%%%%%%%%%%%%%%%%%%%%%%%%%%%%%%%%
%--------------Document Settings--------------%
%%% Preamble (Dokumentenklasse)
% ------------------------------------------------------------------------
% LaTeX - Preambel ******************************************************
% ------------------------------------------------------------------------
% Dokumentklasse (Koma Script)
% ------------------------------------------------------------------------
% basiernd auf www.matthiaspospiech.de/latex/vorlagen Diplomarbeit kompakt
% ========================================================================


\documentclass[%
	11pt,
   headings=big,
   a4paper,          % Papierformat
   %scrreprt,
   %BCOR0mm,          % Bindekorrektur: Zus�tzlicher Rand auf der Innenseite
   %DIV14,            % Seitengr��e (siehe Koma Skript Dokumentation !)
   %1.1headlines,      % Zeilenanzahl der Kopfzeilen
   pagesize,          % Schreibt die Papiergroesse in die Datei.
   oneside,          % Zweiseitiges Layout
   %openright,         % Kapitel beginnen immer auf der rechten Seite
   titlepage,         % Titel als einzelne Seite ('titlepage' Umgebung)  
   %headsepline,       % Linie unter Kolumnentitel ()
   %plainheadsepline, % Linie unter Kolumnentitel () plain Seitenstil
   %nochapterprefix,  % keine Ausgabe von 'Kapitel:'
   %chapterprefix=false,
   %liststotoc,
   %listof=totoc,
   %intoc,
   %bibtotoc,         % Bibliographie ins TOC   
   %bibtotocnumbered, % Bibliographie ins TOC mit Kapitelnummer
   bibliography=totocnumbered,
   %tocindent,        % eingereuckte Gliederung
   %toc=graduated,
   %listsindent,      % eingereuckte LOT, LOF
   %listof=graduated,
   %pointednumbers,   % �berschriftnummerierung mit Punkt, siehe DUDEN !
   %pointlessnumbers, % �berschriftnummerierung ohne Punkt, siehe DUDEN !
   %numbers=noenddot,
   %cleardoubleempty, % Leere linke Seite bei Zweiseitenlayout vor Kapitel
   %cleardoublepage=empty,
   %fleqn,             % Formeln werden linksbuendig angezeigt
   %parindent,        % Absatz mit Einzug (Standard)
   %halfparskip,      % Absatz halbe Zeile Abstand
   %parskip=half,
   %parskip,          % Absatz ganze Zeile Abstand
]{article}            % Klassen: scrartcl, scrreprt, scrbook

%--------------Includes--------------%
\usepackage{amsthm}
\usepackage[utf8]{inputenc}
\usepackage[english,german]{babel}
\usepackage{amsmath}
\usepackage{amsfonts}
\usepackage{amssymb}
\usepackage{xspace} % Define commands that don't eat spaces.
\usepackage{geometry}
\usepackage{fancyhdr}
\usepackage{array}
\usepackage{multirow}
\usepackage{booktabs}
\usepackage{float}
\usepackage{url}
\usepackage{MnSymbol}
\usepackage{graphicx}
\usepackage{epstopdf}
\usepackage{subfig}
\usepackage[hidelinks]{hyperref}



\graphicspath{{fig/Acc_Calib/}}
\graphicspath{{fig/Simulaton/}}
\graphicspath{{fig/Platform/}}

\usepackage[usenames,dvipsnames]{xcolor}
\usepackage{listings}
\lstset{language=C,
	basicstyle=\ttfamily,
	keywordstyle=\color{blue},
	stringstyle=\color{red},
	commentstyle=\color{OliveGreen},
	frame=single,
	tabsize=2,
	showstringspaces=false,
	breaklines=true,
	basicstyle=\tiny,
}


%Change Max Matrix columns to 20 (Default is 10)
\setcounter{MaxMatrixCols}{20}

%3D Plot
\usepackage{verbatim}
\usepackage{tikz}
\usepackage{3dplot} %requires 3dplot.sty to be in same directory, or in your LaTeX installation
\usepackage[active,tightpage]{preview}  %generates a tightly fitting 
\tdplotsetmaincoords{60}{110}

%Format equation numbering
\usepackage{chngcntr}
\counterwithin{equation}{section}


%%% Alle Namen usw. im Titel und im hyperref-Paket
% ------------------------------------------------------------------------
% LaTeX - Preambel ******************************************************
% ------------------------------------------------------------------------
% pre-work
% ========================================================================

% ToDo kennzeichnen
\newcommand{\workTodo}[1]{\textcolor{red}{todo: #1}}

% Für Datum und Zeit in Fusszeile
%\newcommand{\workMarkDateTime}{16. September 2014}
\newcommand{\workMarkDateTime}{\today{} - \thistime\ Uhr}
\newcommand{\workMarkDate}{\today{}}

% Alle Namen werden im Titel und im hyperref-Paket eingetragen

 % <Typ>
\newcommand{\workTyp}{Master Thesis\xspace}

\newcommand{\workOrt}{DLR Oberpfaffenhofen\xspace}

 % <Titel> der Arbeit
 \newcommand{\TitlePageTitel}{Active and Passive Homing Methods Based on Distance Measurements\xspace}
 
\newcommand{\workTitel}{Active and Passive Homing Methods Based on Distance Measurements\xspace}

 % <Studiengang> z.B. Kommunikationstechnik
\newcommand{\workStudiengang}{Mechatronik und Feinwerktechnik\xspace}

 % Hochschulspezifische Angaben
\newcommand{\workHochschule}{Technical University Munich\xspace}
\newcommand{\Studiengang}{Mechatronik und Feinwerktechnik\xspace}
\newcommand{\Studienrichtung}{Appliance Technology\xspace}
\newcommand{\workTUMLogo}{\includegraphics[scale=0.1]{fig/TUM/TUM_Logo.jpg}}

% <Semester> mit Jahr z.B. Sommersemester 2008  
\newcommand{\workSemester}{Winter Semester 2017/2018\xspace}

% <Name> des Studenten
\newcommand{\workNameStudent}{Ingo von Bargen\xspace}

% <Pruefer> Name des prüfenden (betreuenden) Professor an der Hochschule
\newcommand{\WorkSupervisor}{Philipp Lutz, M.Eng\xspace}
\newcommand{\WorkPruefer}{Dr. rer. nat. Rudolph Triebel\xspace}
\newcommand{\secondWorkPruefer}{???\xspace} 


% Nur bei Abschluss-Arbeiten

% <Datum> der Abgabe der Arbeit (Eidesstatliche Erklärung)
\newcommand{\workDatum}{15. February 2018\xspace}

%\makeatletter
%\renewcommand*\env@matrix[1][*\c@MaxMatrixCols c]{%
%  \hskip -\arraycolsep
%  \let\@ifnextchar\new@ifnextchar
%  \array{#1}}
 
%\begin{pmatrix}[cc|c]
%  1 & 2 & 3\\
%  4 & 5 & 9
%\end{pmatrix}

%--------------Own Commands--------------%
\newcommand{\vlinespace}[1]{\vspace*{#1\baselineskip}} % Abstand
\newcommand{\titleemph}[1]{\textbf{#1}} % Hervorheben
\newcommand{\bmath}[1]{\mathbf{#1}}
\newcommand{\bsymb}[1]{\boldsymbol{#1}}

%--------------Geometry--------------%
 \geometry{
 left=30mm,
 right=20mm,
 top=25mm,
 bottom=25mm,
 }
 
 \parindent 0pt % Einrücken einer neuen Zeile verhindern
 
%--------------Footer--------------%
\pagestyle{fancy}
\fancyhf{}
\lhead{\workNameStudent}
\rhead{\workHochschule}
\rfoot{\thepage}

%\documentclass[tikz,border=2mm]{standalone}



%%%%%%%%%%%%%%%%%%%%%%%%%%%%%%%%%%%%%%%%%%%%%%%%%%%%%%%%%%%%%%%%%%%%%%%%%%%
%%%%%%%%%%%%%%%%%%%%%%%%%% START DOCUMENT %%%%%%%%%%%%%%%%%%%%%%%%%%%%%%%%%
%%%%%%%%%%%%%%%%%%%%%%%%%%%%%%%%%%%%%%%%%%%%%%%%%%%%%%%%%%%%%%%%%%%%%%%%%%%
\begin{document}

%--------------Title Page--------------%
\begin{titlepage}
 \rmfamily % Titelseite in seriefenloser Schrift
		
			\begin{center}
				\centering				

				%logo
				\begin{figure}[H]
					\begin{center}
						\workTUMLogo
					\end{center}
				\end{figure}				
				
			\end{center}

		
   \centering
			\vlinespace{3}
			\Large
   {\huge \textbf{\workTitel}}\\
   
   \vlinespace{3}
   {\LARGE \workTyp}\\
   \vlinespace{1}
   
   by\\

   \vlinespace{1}
   \textbf{\workNameStudent\\}
      
	\vlinespace{3}   
   {\LARGE \textbf{\workHochschule\\}
    Master of Robotics, Cognition and Intelligence\\

   }

\vlinespace{4} 

\begin{tabbing}
\hspace{5cm}\=\kill
 Supervision: \> \WorkSupervisor\\
 First Examiner: \> \WorkPruefer\\
 Second Examiner: \> \secondWorkPruefer\\
 Semester: \> \workSemester\\
\end{tabbing} 

     

     
\end{titlepage}
\newpage

%--------------Abstract--------------%
\begin{center}
{\LARGE \textbf{Abstract}}
\end{center}

\vlinespace{3}

{\Large \textbf{\workTitel}}

\vlinespace{1}

MAVs (Micro Aerial Vehicle) are small, ...


\newpage
\begin{center}
{\LARGE \textbf{German Abstract}}
\end{center}

\vlinespace{3}

\selectlanguage{german}

{\Large \textbf{Zustandsschätzung eines ,,Micro Aerial Vehicle'' mit Hilfe einer Multi-Sensor-Datenfusion}}

\vlinespace{1}

MAVs (Micro Aerial Vehicle) sind ...

\selectlanguage{english}
%--------------Table of contents--------------%
\tableofcontents
%\contentsline {section}{\numberline {2}Structure of the Table of Contents}{3}

% Main Document
%-----------------------------------------------------------------
\section{Introduction}

\subsection{Aim and Motivation}

Micro Aerial Vehicles... 


\subsection{Structure of Thesis}

In section ...


%~\cite[Text]{C1,C2} %Citation
%\footnote{Example footnote}

%\begin{tabbing}
%\hspace{20mm}\=\hspace{20mm}\=\kill
% asdf \>  sdf \> asdf \\ 
% sdf \>  asdf \> fasdfasdf
%\end{tabbing} 
%
%\begin{flushleft}
%\renewcommand{\arraystretch}{1.3}
%
%\begin{tabular}{|c|c|c|c|}
%\hline 
%\textbf{eins} & \textbf{zwei} & \textbf{drei} & \textbf{vier} \\[5pt]
%\hline 
%\textbf{zwei} & vier & sechs & acht \\ 
%\hline 
%\textbf{drei} & sechs & neun & zwölf \\ 
%\hline 
%\textbf{vier} & acht & zwölf & sechzehn \\ 
%\hline 
%\textbf{fünf} & zehn & fünfzehn & zwanzig \\ 
%\hline 
%\end{tabular} 
%\end{flushleft}
%
%
%\begin{center}
%\renewcommand{\multirowsetup}{\centering}
%\begin{tabular}{|>{\bfseries}l|r@{\ --\ }l|}
%\hline
%\multirow{3}{1.5cm}{Planet} & \multicolumn{2}{p{3.5cm}|}%
%{\centering Distance from sun \\ (million km)}\\
%\cline{2-3}
%
%& \multicolumn{1}{c|}{Maximum}
%& \multicolumn{1}{c|}{Minimum}\\
%\hline
%Mercury & 69.4 & 46.8\\
%Venus & 109.0 & 107.6\\
%Earth & 152.6 & 147.4\\
%Mars & 249.2 & 207.3\\
%Jupiter & 817.4 & 741.6\\
%Saturn & 1512.0 & 1346.0\\
%Uranus & 3011.0 & 2740.0\\
%Neptune & 4543.0 & 4466.0\\
%Pluto & 7346.0 & 4461.0\\
%\hline
%\end{tabular}
%\end{center}


\section{Mathematical Premise}
\label{sec:MathPremise}





%...

\section{Summary}
\label{sec:Summary}

The aim of this thesis was ...
\include{Chapters/Appendix}



\listoffigures
\addcontentsline{toc}{section}{List of Figures}

\section*{List of Abbreviations}
\addcontentsline{toc}{section}{List of Abbreviations}

\begin{tabbing}
\hspace{2cm}\=\kill
 %MAV \> Micro Arial Vehicle\\
 %GPS \> Global Positioning System\\

 
\end{tabbing} 

\section*{List of Symbols}
\addcontentsline{toc}{section}{List of Symbols}

\begin{tabbing}
\hspace{1cm}\=\kill
 %$\bmath{\Phi}$ \> Discrete state transition matrix \\
\end{tabbing}


% Bibliography
%-----------------------------------------------------------------
\addcontentsline{toc}{section}{Bibliography}
\begin{thebibliography}{99}

%\bibitem{C1} 25.12.14, \url{http://en.wikipedia.org/wiki/Kalman_filter}

\bibitem{C2} Norbert Schmitz, \emph{Satellitenortung und Sensorfusion zur Lokalisierung von Fahrzeugen in unstrukturierter Umgebung}, Diplomarbeit, (2005)

\bibitem{C3} Murali Rajaa Rajamani, \emph{Data-based Techniques to Improve State Estimation in Model Predictive Control}, Dissertation, 2007

\bibitem{C4} Christian Ascher, \emph{Modulares multisensorielles Indoor Navigationssystem}, Book

\bibitem{WENDEL} Jan Wendel, \emph{Integrierte Navigationssysteme - Sensordatenfusion, GPS und Inertiale Navigation, 2. Auflage}

\bibitem{KFWRTA} Charles K. Chui, Guanrong Chen, \emph{Kalman Filtering: with Real-Time Applications}

\bibitem{Maybeck} Maybeck, \emph{Stochastic Models: Estimation and Control: v. 1}

\bibitem{ComSens2} Schmid, K. and Hirschmüller, H. (2013) \emph{Stereo vision and imu based real-time ego-motion and depth image computation on a handheld device} In IEEE International Conference on Robotics and Automation (ICRA), Karlsruhe, Germany.

\bibitem{C5} 27.02.15, \url{http://www.gpo.gov/fdsys/pkg/CFR-2005-title47-vol4/pdf/CFR-2005-title47-vol4-sec73-208.pdf}

\bibitem{ublox_website} 03.01.15, \url{http://www.u-blox.com/en/gps-modules/pvt-modules/neo-m8.html}

\bibitem{MPU6050_pic} 09.01.15, \url{http://www.invensense.com}

\bibitem{GPS_module_pic} 09.01.15, \url{http://www.drotek.fr/shop/511-1613-thickbox_dm/ublox-neo-m8-gps-module.jpg}

\bibitem{PX4FLOW_pic} 	09.01.15, \url{https://pixhawk.org/_detail/modules/px4flow-bottom.jpg?id=modules3Apx4flow}

\bibitem{MS5611_pic} 	09.01.15, \url{http://www.amsys.de/products/ms5611.htm}

\bibitem{HMC5883L} 	05.02.15, \url{https://www.sparkfun.com/products/10494}

\bibitem{atmosp_physics} 15.03.15, \url{http://www.atmosp.physics.utoronto.ca/PHY2509/on9901_ch5.pdf}

\bibitem{ComSens1} 19.03.15, \url{https://pixhawk.org/modules/px4flow}

\bibitem{ComSens3} 19.03.15, \url{http://www.velodynelidar.com}

\bibitem{ComSens4} 19.03.15, \url{http://www.ar-tracking.com/home/}

\bibitem{ComSens5} 19.03.15, \url{http://www.adafruit.com/product/164}

\bibitem{ComSens6} 19.03.15, \url{http://www.lightware.co.za/shop/en/4-laser-rangefinders}
\end{thebibliography}


%--------------Acklnowlegements--------------%
\vlinespace{10}

\begin{center}
{\LARGE \textbf{Acknowledgements}}
\end{center}

\vlinespace{3}

I would especially like to thank ...

\vlinespace{1}

I am also very grateful for the valuable advice ...
\newpage

%--------------Erklärung--------------%
\begin{tabbing}
\hspace{3cm}\=\kill
 Name: \> Ingo von Bargen\\
 geb.: \> 25.11.1991\\
 Matr.Nr.: \> 03667417\\
 Studiengang: \> Robotic, Cognition, Intelligence\\
 Semester: \> WS 2017/18\\
\end{tabbing}

\vlinespace{2}

{
\begin{center}
{\huge Erklärung}\\
\vlinespace{1}
gemäß § 13 Abs. 5 RaPO
\end{center}
}

\vlinespace{4}

Hiermit erkläre ich, dass ich die Masterarbeit mit dem Titel ,,\workTitel'' selbstständig verfasst, noch nicht anderweitig für Prüfungszwecke vorgelegt, keine anderen als die angegebenen Quellen oder Hilfsmittel benützt
sowie wörtliche und sinngemäße Zitate als solche gekennzeichnet habe.

\vlinespace{12}


\_\_\_\_\_\_\_\_\_\_\_\_\_\_\_\_\_\_\_\_\_\_\_\_\_\_\_\_\quad \quad \quad \quad \quad \quad \quad \quad \quad \quad \  \_\_\_\_\_\_\_\_\_\_\_\_\_\_\_\_\_\_\_\_\_\_\_\_\_\_\_\_

Ort, Datum \quad \quad \quad \quad \quad \quad \quad \quad \quad \quad \quad \quad \quad \quad \quad Unterschrift


\end{document}
